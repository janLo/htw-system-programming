
\navone{Allgemeines}
\navtwo{Aufgabe}
Entwickeln Sie, in C, C++ (oder Perl), einen "`benutzerfreundlichen"' und "`robusten"' SMTP-Server.

Der Server soll den Clients \"{u}ber eine Socketverbindung (freier Port) die SMTP-Kommandos in einer einfachen, verst\"{a}ndlichen Form bereitstellen/anfordern, die Eingaben pr\"{u}fen, zusammenstellen und zum Senden der Mail an einen SMTP-Server (z.B. mail.informatik.htw-dresden.de, Postfix) schicken. 

Dazu ist ebenfalls eine eigene Socketverbindung zu nutzen, die Nutzung von Kommandos wie mail oder sendmail ist nicht zul\"{a}ssig. Dies ist auch nicht m\"{o}glich, da der Client, im Sinne der Nutzerfreundlichkeit, eine Sendequittung (zum Test ev. gesamtes Protokoll), erhalten soll. Ob die Kommandos als ein gesamter String oder einzeln gesendet werden, ist freigestellt.

Als Client kann telnet genutzt werden.

Der Server soll gleichzeitig mit mehreren Clients arbeiten k\"{o}nnen, daf\"{u}r sind Threads zu nutzen.

\navtwo{Herangehensweise}
Die Aufgabenstellung beschreibt ein simples SMTP Relay, welches E-Mails per SMTP annimmt und einfach weiterleitet. Das macht es n\"{o}tig, sowohl die Grundfunktionalit\"{a}ten eines SMTP Servers(Annahme), als auch die eines Clients (Forward) zu implementieren.

Bei der weiteren Analyse der Aufgabe wurde festgestellt, das diese bei der Implementierung in 4 Module aufgeteilt werden kann:
\begin{description}
  \item[smtprelay]
    Dies beherbergt die \texttt{main} Funktion, k\"{u}mmert sich um das parsen der Kommandozeilenoptionen, inkl. nachschlagen der Hostnamen. Zudem ist es f\"{u}r die Fehlerausgabe verantwortlich. (Listing \appref{smtprelay.c}, \appref{smtprelay.h}) 
  \item[connection] 
    Dies ist f\"{u}r die komplette Verbindungsverwaltung inkl. der Erzeugung neuer Client Threads zust\"{a}ndig (Listing \appref{connection.c}, \appref{connection.h})
  \item[session] 
    Dieses Modul implementiert eine Client-Session. Es beherbergt den kompletten Code zur Kommunikation mit einem Client. (Listing \appref{session.c}, \appref{session.h})
  \item[sender] 
    Das sender Modul ist f\"{u}r die Kommunikation mit dem Remoteserver zust\"{a}ndig, an welchen die E-Mails weitergeleitet werden. (Listing \appref{sender.c}, \appref{sender.h})
\end{description}

Das Ziel war, ein einzelnes Binary zu schaffen, welches mittels Kommandozeilen-Optionen konfigurierbar ist und die komplette, in der Aufgabenstellung geforderte, Funktionalit\"{a}t abdekt.

\navtwo{Build}\label{build}
Zu dem entstandenen Programm existiert ein Makefile (Listing \myref{Makefile}{Makefile}), welches vier wichtige Targets kennt:
\begin{description}
  \item[binary] Baut das normale Programm, hier sind die Ausgaben w\"{a}hrend der Laufzeit auf kritische Fehler reduziert. 
  \item[debug] Baut eine debug Version des Programmes, welche w\"{a}hernd der Laufzeit besonders viele Informationen ausgeben kann .
  \item[all] Baut sowohl die debug Version des Programms, als auch die normale, nicht-debug Version.
  \item[clean] L\"{o}scht alle Object-Dateien und die erzeugten Binarys. 
\end{description}
Das zu erstellende Target kann mittels \texttt{make <target>} erzeugt werden. Wird hinter \texttt{make} kein Target angegeben, so wird \texttt{make all} ausgef\"{u}hrt.



\navone{Nutzung}
\navtwo{Server}
Der Server ist ein einfaches Programm, welchen auf der Kommandozeile ausgef\"{u}hrt werden kann.

Er wird komplett \"{u}ber die Kommandozeile konfiguriert und kennt dabei folgende Optionen:
\begin{description}
  \item[\texttt\bf{-q}] Schaltet in den quiet Mode, in welchem die Fehlerausgaben ausgeschaltet sind.
  \item[\texttt\bf{-v}] Schaltet den verbose Mode ein, in welchem besonders viele Informationen ausgegeben werden. Diese Option ist jedoch nur in der debug Version vorhanden (siehe \myref{build}{Build})
  \item[\texttt\bf{-a <bind\_address>}] Mit diesem Schalter wird die Adresse angegeben, an die der Server gebunden werden soll. Um die standartm\"{a}ßig eingesttelle Adresse zu erfahren f\"{u}hren Sie bitte das Programm mit dem Schalter \texttt{-h} aus. 
  \item[\texttt\bf{-p <bind\_port>}] Mit diesem Schalter geben Sie den Port an, an den der Server gebunden werden soll. Auch hier wird der Standardwert durch den Schalter \texttt{-h} ausgegeben.
  \item[\texttt\bf{-d <remote\_address>}] Hiermit kann die Adresse des Remoteservers, an den alle eingehenden E-Mails weitergeleitet werden, angegeben werden. Der Standartwert wird mit dem Schalter \texttt{-h} ausgegeben.
  \item[\texttt\bf{-r <remote\_port>}] Dieser Schalter dient zur Eingabe des Remoteports, also des Ports, an dem der Remoteserver lauscht. Standardwerte sind mit \texttt{-h} zu erfragen.
  \item[\texttt\bf{-h}] Gibt einen Hilfetext mit allen Standartwerten aus. 
\end{description}
Adressen die \texttt{<bind\_address>} k\"{o}nnen dabei sowohl als IP Adresse, als auch als Hostname angegeben werden.

Wurde das Programm korrekt ausgef\"{u}hrt, so lauscht es auf dem angegebenem Port und wartet auf Clients.

Der Remoteserver muss ein SMTP Server sein, welcher Mindestens die Kommandos \texttt{HELO}, \texttt{MAIL FROM}, \texttt{RCPT TO}, \texttt{DATA} und \texttt{QUIT} versteht. Der Datenblock wiederum muss mit \texttt{<CR><LF>.<CR><LF>} abschließbar sein.

\navtwo{Client}
Der Client muss die wichtigsten SMTP Kommandos beherrschen. Implementiert sind:
\begin{description}
  \item[\texttt\bf{HELO <client\_name>}] Anmeldung des Clients.
  \item[\texttt\bf{MAIL FROM:<mail\_addr>}] Angabe des Absenders.
  \item[\texttt\bf{RCPT TO:<mail\_addr>}] Angabe des Empf\"{a}ngers.
  \item[\texttt\bf{DATA}] Beginn des Datenblocks (Ende mit \texttt{<CR><LF>.<CR><LF>}).
  \item[\texttt\bf{QUIT}] Beenden der Verbindung.
  \item[\texttt\bf{RSET}] R\"{u}cksetzten der Verbindung bis nach \texttt{HELO}. 
  \item[\texttt\bf{NOOP}] Gibt immer eine 250 zur\"{u}ck (zum Bsp. zum Testen der Verbindung).
\end{description}

Dabei muss eine Normale Sitzung immer nach dem SMTP-Protokoll (siehe \myref{rfc2821}{rfc2821}) ablaufen. Das heißt, das zuerst immer ein \texttt{HELO} kommen muss, dann beliebig viele Mailbl\"{o}cke, wobei ein Mailblock aus \texttt{HELO}$\rightarrow$\texttt{RCPT TO}$\rightarrow$\texttt{DATA}$\rightarrow$Datenblock mit abschließendem \texttt{<CR><LF>.<CR><LF>} besteht, und abschließend ein \texttt{QUIT}.

Die Replycodes des Servers sind entsprechend dem SMTP Standard (siehe \myref{rfc2821}{rfc2821}) implementiert.

\navone{Programmdokumentation}
\navtwo{Modul smtprelay}\label{mod:smtprelay}
\navthree{Beschreibung}
Hier sind alle Funktionen definiert, die vor der \"{O}ffnung des Sockets aufzurufen sind. Außerdem ist die die Fehlerausgabe, sowie die debug Ausgabe (im Header) definiert. 

Die Debugausgabe ist durch mehrere Makros realisiert, welche nur dann einen Effekt haben, wenn die debug Variante des Programmes gebaut wird (siehe \myref{build}{Build}). Die Makros haben alle unterschiedliche Argumente, je nachdem was ausgegeben werden soll. Zudem sind sie noch unterteilt in normale und DEBUG\_CLNT Makros, welche zus\"{a}tzlich noch die aktuelle Threadid ausgeben.

Die Quellen sind in den Listings \appref{smtprelay.c} und \appref{smtprelay.h} aufgef\"{u}hrt.

\navthree{Funktion main}\label{fn:main}
Die Main Funktion ist die Funktion, welche im Programm als erstes aufgerufen wird. Sie stellt den Startpunkt des Programms dar.

Hier werden zuerst die Interupt Handler installiert, welche daf\"{u}r sorgen, dass der Server Socket auch ordnungsgem\"{a}ß geschlossen wird. Anschließend wird die Funktion \myref{fn:process_opt}{process\_opt} zum verarbeiten der Kommandozeilenargumente aufgerufen und \myref{fn:create_conn}{create\_conn} aus \myref{mod:connection}{connection} aufgerufen um den Socket zu kreieren.

Wenn diese Funktion zur\"{u}ckkehrt wird das Programm beendet.

\navthree{Funktion process\_opt}\label{fn:process_opt}
Diese Funktion wertet mittels \texttt{getopt} und den Funktionen \myref{fn:host_opt}{host\_opt} und \myref{fn:port_opt}{port\_opt} die Kommandozeilenargumente aus, schreibt diese in die globale Variable \texttt{app}, in der alle Konfigurationsdaten gehalten werden und ruft bei Bedarf die Funktion zur Aufgabe der Hilfe \myref{fn:write_help}{write\_help} auf.

Die R\"{u}ckgabe ist im Erfolgsfall \texttt{OK}, sonst \texttt{FAIL} (beide Werte definiert in Listing \appref{smtprelay.h})

\navthree{Funktion write\_help}\label{fn:write_help}
Diese Funktion dient dazu, eine Hilfe zur Programmbenutzung auszugeben.

\navthree{Funktion host\_opt}\label{fn:host_opt}
Diese Funktion dient dazu, die Addressargumente zu pr\"{u}fen und umzuwandeln.

Das Argument \texttt{buff} beinhaltet dabei das Kommandozeilenargument und \texttt{err} eine Fehlermeldung, welche ausgegeben werden soll, wenn das Argument nicht valide ist.

Wird festgestellt, das das Argument eine IP Adresse war, so wird diese in einen neuen Speicherbereich kopiert und dessen Adresse zur\"{u}ckgegeben.

Ansonsten wird versucht, das Argument als Hostnamen zu nutzen und per \texttt{gethostbyname} die zugeh\"{o}rige IP Adresse zu ermitteln. Gelingt dies, so wird die IP Adresse als String in einen neuen Speicherbereich geschrieben und dessen Adresse zur\"{u}ckgegeben.

Sollte es nicht m\"{o}glich sein, eine IP Adresse zu ermitteln, so wird NULL zur\"{u}ckgegeben.
\navthree{Funktion port\_opt}\label{fn:port_opt}
Diese Funktion dient dazu eine per Kommandozeile \"{u}bergebene Portnummer auf G\"{u}ltigkeit zu pr\"{u}fen.

Das Argument \texttt{buff} beinhaltet dabei das Kommandozeilenargument und \texttt{err} eine Fehlermeldung, welche ausgegeben werden soll, wenn das Argument nicht valide ist.

Ist die \"{u}bergebene Option eine valide Portnummer, dann wird diese in einen neuen Speicherbereich geschrieben und dessen Adresse zur\"{u}ckgegeben, sonst NULL.

\navthree{Funktion put\_err}\label{fn:put_err}
Diese Funktion dient zur Ausgabe eines Fehlers. Dabei wird zuerst das \"{u}bergebene Argument inklusive der in \texttt{ERROR\_PREF} definierten Formatierung und anschließend der zu \texttt{errno} geh\"{o}rende Fehlertext mittels \texttt{perror} auf \texttt{stderr} ausgegeben.

\navthree{Funktion put\_err\_str}\label{fn:put_err_str}
Diese Funktion gibt einfach den im Argument \"{u}bergebenen Fehlertext inklusive der in \texttt{ERROR\_PREF} definierten Formatierung auf \texttt{stderr} ausgegeben.

\navtwo{Modul connection}\label{mod:connection}
\navthree{Beschreibung}
Das Modul connection beherbergt alle Funktionen, die sich mit den Socketverbindungen besch\"{a}ftigt. Dies bezieht sich sowohl auf die Verbindungen zu den Clients, als auch der zum remoteserver.

Zudem werden hier die neuen Threads f\"{u}r die Clients erzeugt, da jeder Client seinen eigenen Thread bekommt. Auch der Interupt handler ist hier definiert, da in diesem die Socketverbindung des Servers beendet werden muss.

Die Quellen sind in den Listings \appref{connection.c} und \appref{connection.h} aufgef\"{u}hrt.




\navthree{Funktion create\_socket}\label{fn:create_socket}
Dies erzeugt einen neuen Inet Socket, welcher hier auch gleich an die richtige Adresse gebunden wird.

\navthree{Funktion process\_client}\label{fn:process_client}
Dies ist die erste Funktion, die im neuen Thread aufgerufen wird.

Sie startet mittels \myref{fn:start_session}{start\_session} die Client Session, und schließt den Client Socket wenn die Session beendet wurde.

\navthree{Funktion wait\_connect}\label{fn:wait_connect}
Hier wird auf Client Verbindungen gewartet. 

Kommt solch eine zustande, so wird ein neuer Thread gestartet und in diesem die Funktion \myref{fn:process_client}{process\_client} zum starten einer neuen Sitzung aufgerufen.

Der neue Thread wird gleich mittels \texttt{pthread\_detach} detacht, um Speicherl\"{o}chern durch nicht aufger\"{a}umte Threads vorzub\"{a}ugen.

\navthree{Funktion create\_conn}\label{fn:create_conn}
Diese Funktion dient zum \"{o}ffnen des Socket. Sie bekommt Hostname und Port zum binden \"{u}bergeben und generiert daraus eine \texttt{sockaddr\_in}.

Dann wird der Socket mit \myref{fn:create_socket}{create\_socket} ge\"{o}ffnet und mit \myref{fn:wait_connect}{wait\_connect} auf Clients gewartet.

Am Ende wird mit \myref{fn:quit_conn}{quit\_conn} die Verbindung beendet (sollte nie passieren, da das Warten nur durch ein Signal oder einen Fehler unterbrochen wird).


\navthree{Funktion quit\_conn}\label{fn:quit_conn}
Diese Funktion schließt die als Filedescriptor \"{u}bergebene Socketverbindung.


\navthree{Funktion create\_remote\_conn}\label{fn:create_remote_conn}
Diese Funtion baut eine Verbindung zu dem Remoteserver auf, an den die Mails weitergeleitet werden sollen. Zur\"{u}ckgegeben wird der Filedescriptor des Sokets, \texttt{-1} sonst.

Die Verbindungsdaten (IP Adresse, Portnummer) werden als Argumente an die Funktion \"{u}bergeben.

\navthree{Funktion quit\_remote\_conn}\label{fn:quit_remote_conn}
Diese Funktion schließt die als Filedescriptor \"{u}bergebene Socketverbindung.

\navthree{Funktion sig\_abrt\_conn}\label{fn:sig_abrt_conn}
Dies ist der Signal Handler, der aufgerufen wird, wenn eines der Signale \texttt{SIGABRT}, \texttt{SIGTERM}, \texttt{SIGQUIT} oder \texttt{SIGINT} eintifft. 

Sie schließt den Server Socket und beendet das Programm.





\navtwo{Modul session}\label{mod:session}
\navthree{Beschreibung}
Im Modul session sind die Funktionen definiert, die zur Kommunikation mit dem Client n\"{o}tig sind. 

Dazu geh\"{o}rt die Ein- und Ausgabe auf dem Client Socket, die Abarbeitung der im SMTP Protokoll definierten Kommandosequenzen sowie das Testen der Eingaben des Clients.

Die Quellen sind in den Listings \appref{session.c} und \appref{session.h} aufgef\"{u}hrt.

\navthree{Funktion write\_client\_msg}\label{fn:write_client_msg}
Diese Funktion schreibt eine Ausgabe auf den Socket. Die Nachricht ist ein im Header des Modules \appref{session.h} definierter Formatstring (\texttt{MSG\_*}), in den der als Argument \"{u}bergebene Replycode, sowie ein optionales String Argument eingebaut wird.

Zur\"{u}ckgegeben wird bei Erfolg \texttt{OK}, sonst \texttt{FAIL} (definiert in \appref{smtprelay.h}).

\navthree{Funktion check\_addr}\label{fn:check_addr}
Diese Funktion \"{u}berpr\"{u}ft eine vom Client angegebene Hostadresse. 

Da dieses Programm lediglich ein Relay sein soll, welches keine Mails ausliefern soll ist es nicht n\"{o}tig, des Hostnamen auf Existenz zu pr\"{u}fen. Es wird lediglich gepr\"{u}ft, ob min. zwei Zeichen eingegeben wurden. 

Die genauere \"{U}berpr\"{u}fung wird dann dem Remote Server \"{u}berlassen.

Ist zu einem sp\"{a}terem Zeitpunkt eine komplexere Pr\"{u}fung n\"{o}tig, so ist dies an dieser Stelle sehr einfach Implementierbar.

\navthree{Funktion check\_mail}\label{fn:check_mail}
Diese Funktion ist, analog zu \myref{fn:check_addr}{check\_addr} f\"{u}r das Testen einer eingegebenen Mailadresse zust\"{a}ndig. 

Dabei wird einfach gepr\"{u}ft, ob eine mindestens 3 Zeichen lange Zeichenkette, gefolgt von einem @ und einem Hostnamen, welcher die \"{U}berpr\"{u}fung in Funktion \myref{fn:check_addr}{check\_addr} besteht, eingegeben wurde.

Eine genauere \"{U}berpr\"{u}fung wird auch hier dem Remote Server \"{u}berlassen, ist aber an dieser Stelle einfach implementierbar.

\navthree{Funktion check\_input}\label{fn:check_input}
Diese Funktion testet eine vom Client stammende gelesene Zeile.

Dazu bekommt sie die gelesene Zeile, ein SMTP Kommando, das erwartet wird, ein Trennzeichen, mit dem das Kommando vom restlichem String getrennt ist, eine Funktion, mit der das Argument des Kommandos \"{u}berpr\"{u}ft wird und eine Adresse, an die das Argument geschrieben werden soll \"{u}bergeben.

Zuerst wird \"{u}berpr\"{u}ft, ob das Trennzeichen im String zu finden ist. Ist dies nicht der Fall, so wird \texttt{CHECK\_DELIM} zur\"{u}ckgegeben.

Anschließend wird getestet, ob der gelesene String das erwartete Kommando ist, wenn nicht wird \texttt{CHECK\_PREF} zur\"{u}ckgegeben.

Dann wird gepr\"{u}ft, ob das Trennzeichen ein nicht-Nullterminierungszeichen war. Im Erfolgsfall wird getestet, ob die L\"{a}nge des Arguments gr\"{o}ßer 0 ist (sonst wird \texttt{CHECK\_ARG} zur\"{u}ckgegeben) und wenn ja, ob die Testfunktion (wenn angegeben) einen positiven R\"{u}ckgabewert zur\"{u}ckgibt. Ist dies nicht der Fall, wird die Funktion mit \texttt{CHECK\_ARG} oder \texttt{CHECK\_ARG\_MSG} beendet.

Wurden alle Tests passiert, so wird das Argument in einen neuen Speicherbereich geschrieben und \texttt{CHECK\_OK} zur\"{u}ckgegeben.

\navthree{Funktion fetch\_input\_line}\label{fn:fetch_input_line}
Diese Funktion holt \"{u}ber den Socket eine Zeile vom Client. Als Argumente bekommt sie den Filedescriptor der Socketverbindung, ein SMTP Kommando, das erwartet wird, ein Trennzeichen, mit dem das Kommando vom restlichem String getrennt ist, eine Funktion, mit der das Argument des Kommandos \"{u}berpr\"{u}ft wird und eine Adresse, an die das Argument geschrieben werden soll \"{u}bergeben.

Zuerst wird die Zeile vom Socket gelesen und diese mittels \myref{fn:check_input}{check\_input} gepr\"{u}ft, ob es sich um die erwartete Zeile handelt. Ist dies der Fall, so wird \texttt{READ\_OK} zur\"{u}ckgegeben. Schl\"{a}gt die \"{U}berpr\"{u}fung des Argumentes in \myref{fn:check_input}{check\_input} fehl, so wird ein fehler an den Client geschickt und erneut auf eine Zeile gewartet.

Handelt es sich bei der gelesenen Zeile nicht um die erwartete, so wird (ebenfalls mit \myref{fn:check_input}{check\_input}) gepr\"{u}ft, ob es sich um ein \texttt{RSET}, \texttt{QUIT}, \texttt{NOOP} oder eines der nicht implementierten Kommandos \texttt{VRFY}, \texttt{EXPN} bzw. \texttt{HELP} handelt. Zum Schluss wird noch getestet, ob es sich um ein anderes SMTP Kommando kandelt, welches an dieser Stelle nicht erwartet wurde.
\begin{itemize}
  \item Bei \texttt{RSET} wird eine Meldung ausgegeben, das erfolgreich zur\"{u}ckgesetzt wurde und \texttt{READ\_RESET} zur\"{u}ckgegeben. 
  \item Bei \texttt{QUIT} wird eine Abschiedsnachricht an den Client gesandt und \texttt{READ\_QUIT} zur\"{u}ckgegeben. 
  \item Bei \texttt{NOOP} Wird eine kurze Nachricht an den Client gesandt und auf die n\"{a}chste Zeile gewartet.
  \item Bei einem der unimplementierten Kommandos wird eine Meldung an den Client geschickt, dass dieses Kommando nicht implementiert ist und auf die n\"{a}chste Zeile gewartet.
  \item Bei einem unerwartetem Kommando wird ausgegeben, dass es sich um eine  falsche Kommandosequenz handelt und auf die n\"{a}chste Zeile gewartet.
  \item Wenn die Eingabe keinem Kommando zugeordnet werden konnte wird eine Fehlermeldung an den Client geschickt, dass die Syntax falsch sei und auf die n\"{a}chste Zeile gewartet. 
\end{itemize}

Bei Lesefehlern auf dem Socket wird \texttt{READ\_ERR} zur\"{u}ckgegeben und ein Fehler ausgegeben.

\navthree{Funktion read\_data}\label{fn:read_data}
Diese Funktion dient dazu, den Datenblock einer Mail zeilenweise zu lesen und daraus eine Liste vom Typ \texttt{data\_line\_t} zu generieren.

Sie \"{u}berpr\"{u}ft die Eingabe auch auf das Terminierungszeichen \texttt{<CR><LF>.<CR><LF>} und beendet an dieser Stelle den Datenblock.


\navthree{Funktion session\_sequence}\label{fn:session_sequence}
Diese Funktion dient dazu, eine Mail Sequenz (\texttt{MAIL FROM} $\rightarrow$ \texttt{RCPT TO} $\rightarrow$ \texttt{DATA} $\rightarrow$ Daten $\rightarrow$ \texttt{<CR><LF>.<CR><LF>}) abzuarbeiten.

Dazu wird mit \myref{fn:fetch_input_line}{fetch\_input\_line} auf das entsprechende Kommando bzw. mit \myref{fn:read_data}{read\_data} auf Daten gewartet und jeweils eine Statusnachricht zur\"{u}ckgegeben.

Bei Fehlern wird mit \texttt{SESSION\_ABORT} abgebrochen. Ein empfangenes \texttt{QUIT} l\"{a}sst die Funktion ein \texttt{SESSION\_QUIT} und \texttt{RSET} ein \texttt{SESSION\_RESET} zur\"{u}ckgeben. Bei erfolgreich abgearbeiteter Mail Sequenz wird \texttt{SESSION\_SEND} zur\"{u}ckgegeben.

\navthree{Funktion put\_forward\_proto}\label{fn:put_forward_proto}
Diese Funktion gibt das Protokoll vom versenden der E Mail an den Remoteserver als Replycode zeilenweise aus.


\navthree{Funktion start\_session}\label{fn:start_session}
Dies ist die Funktion, die eine Mailsession f\"{u}r einen Client startet. Sie wird dabei vin \myref{fn:process_client}{process\_client} aus dem Modul \myref{mod:connection}{connection} aufgerufen.

Sie begr\"{u}ßt den Client, wartet auf sein \texttt{HELO}, startet mit \myref{fn:session_sequence}{session\_sequence} die Mail Sequenzen, \"{u}bernimmt die Behandlung der R\"{u}ckgabewerte (r\"{u}cksetzten, beenden oder abbrechen der Verbindung) und schickt mittels \myref{fn:forward_mail}{forward\_mail} die Mail schließlich an den Remoteserver.



\navtwo{Modul sender}\label{mod:sender}
\navthree{Beschreibung}
Das Modul sender definiert alle Funktionen zur Kommunikation mit dem Remoteserver.

Dazu geh\"{o}rt das Senden der Kommandos, das auswerten der Replys sowie das mitschreiben eines Protokolls in Form einer Liste vom Typ \texttt{data\_list\_t}, welches dem Client dann in dem abschließendem Reply ausgegeben wird.

Die Quellen sind in den Listings \appref{sender.c} und \appref{sender.h} aufgef\"{u}hrt.

\navthree{Funktion new\_proto\_entry}\label{fn:new_proto_entry}
Diese Funktion generiert aus der \"{u}bergebenen Zeichenkette ein neues Element vom Typ \texttt{data\_list\_t} f\"{u}r das Protokoll.

\navthree{Funktion wind\_proto}\label{fn:wind_proto}
Diese Funktion bekommt einen Pointer vom Typ \texttt{data\_line\_t}, welcher auf einen beliebigen Protokolleintrag zeigt und gibt einen Pointer vom gleichem Typ zur\"{u}ck, der auf das letzte Element des Protokolls zeigt.

\navthree{Funktion read\_remote}\label{fn:read_remote}
Dies ist eine Wrapperfunktion f\"{u}r \texttt{read}, die zus\"{a}tzlich die Fehlerausgabe und das mittscreiben des Protokolls \"{u}bernimmt.

\navthree{Funktion write\_remote}\label{fn:write_remote}
Dies ist eine Wrapperfunktion f\"{u}r \texttt{write}, die zus\"{a}tzlich die Fehlerausgabe und das mittscreiben des Protokolls \"{u}bernimmt.

\navthree{Funktion free\_protocol}\label{fn:free_protocol}
Die Funktion \texttt{free\_protocol} bekommt den Pointer auf das Protokoll \"{u}bergeben und gibt dessen Speicher Eintrag f\"{u}r Eintrag wieder frei.

\navthree{Funktion extract\_status}\label{fn:extract_status}
Diese Funktion extrahiert aus einem, vom Remoteserver, empfangenem String den Replycode und gibt ihn als \texttt{int} zur\"{u}ck.

\navthree{Funktion try\_command}\label{fn:try_command}
Diese Funktion sendet ein SMTP Kommando mittels \myref{fn:write_remote}{write\_remote} an den Remoteserver und wartet die Antwort ab: 
\begin{itemize}
  \item Ist die ampfangene Antwort wie erwartet, so wird \texttt{COMMAND\_OK} zur\"{u}ckgegeben. 
  \item Ist der empfangene Replycode ein \texttt{4xx} Code, so wird (max. \texttt{SEND\_MAXTRY} mal) versucht das Kommando erneut zu senden. 
  \item Ist der Replycode ein anderer oder \texttt{SEND\_MAXTRY} Versuche schief gegangen, so wirf \texttt{COMMAND\_FAIL} zur\"{u}ckgegeben.
\end{itemize}
Zudem werden Lese und schreibfehler abgefangen.


\navthree{Funktion send\_mail}\label{fn:send_mail}
Diese Funktion schickt die im SMTP Protokoll (siehe \myref{rfc2821}{rfc2821}) definierte Kommandosequenz (\texttt{HELO} $\rightarrow$ \texttt{MAIL FROM} $\rightarrow$ \texttt{RCPT TO} $\rightarrow$ \texttt{DATA} $\rightarrow$ Daten $\rightarrow$ \texttt{<CR><LF>.<CR><LF>} $\rightarrow$ \texttt{QUIT}) zum versenden der E Mail an den Remoteserver.

Dazu benutzt sie vor allem die Funktionen \myref{fn:try_command}{try\_command} und \myref{fn:write_remote}{write\_remote}.

Nach erfolgreichem Versand wird \texttt{OK} zur\"{u}ckgegeben, ansonsten \texttt{FAIL} (definiert in \appref{smtprelay.h})

\navthree{Funktion forward\_mail}\label{fn:forward_mail}
Diese Funktion startet das verschicken der E Mail an den Remoteserver. Aufgerufen wird sie von \myref{fn:start_session}{start\_session} aus dem Modul \myref{mod:session}{session}.

Sie l\"{a}sst zuerst von \myref{fn:create_remote_conn}{create\_remote\_conn} aus dem Modul \myref{mod:connection}{connection} eine Verbindung zum Remoteserver aufbauen, startet dann durch einen Aufruf von \myref{fn:send_mail}{send\_mail} das senden der Mail und l\"{a}sst anschließend mit \myref{fn:put_forward_proto}{put\_forward\_proto} das Protokoll ausgeben. 

Der Replycode wird dabei abh\"{a}ngig vom Versendeerfolg ausgegeben.




